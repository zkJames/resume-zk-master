% !TEX TS-program = xelatex
% !TEX encoding = UTF-8 Unicode
% !Mode:: "TeX:UTF-8"

\documentclass{resume}
\usepackage{zh_CN-Adobefonts_external} % Simplified Chinese Support using external fonts (./fonts/zh_CN-Adobe/)
% \usepackage{NotoSansSC_external}
% \usepackage{NotoSerifCJKsc_external}
% \usepackage{zh_CN-Adobefonts_internal} % Simplified Chinese Support using system fonts
\usepackage{linespacing_fix} % disable extra space before next section
\usepackage{cite}

\begin{document}
\pagenumbering{gobble} % suppress displaying page number

\name{朱奎\quad James}

\basicInfo{
  \email{zhukui.1998@qq.com} \textperiodcentered\ 
  \phone{(+86) 131-2671-6166} \textperiodcentered\ 
  \github{https://github.com/zkJames}\
}
\basicInfo{
  \faHome 山东 潍坊\:\textperiodcentered\:\
  \faLaptop 后端开发、数据工程\:\textperiodcentered\:\
  \faMars 男
}
  % \linkedin[billryan8]{https://www.linkedin.com/in/billryan8}}
  \section{\faGraduationCap\ 教育背景}
  \datedsubsection{\textbf{中国科学院大学},计算机软件与理论}{2021.9 - 预计2024.6毕业}
  \begin{itemize}
    \item \textit{学术硕士在读(保研)}
    \item \textbf{GPA:} 3.5 / 4.0\space \space \textbf{排名:} 4 /  63 (前10\%)
  \end{itemize}
  \datedsubsection{\textbf{河南科技大学},计算机科学与技术}{2017.9 - 2021.6}
  \begin{itemize}
    \item \textit{本科}
    \item \textbf{GPA:} 4.5 / 5.0\space \space \textbf{排名:} 3 / 144 (前3\%)
  \end{itemize}

  \section{\faUsers\ 实习经历}

  \datedsubsection{\textbf{阿里巴巴(中国)有限公司}}{2023年6月 -- 至今}
  \role{直连交易平台Java后端开发实习生}{飞猪·住宿行业研发中心·行业基础平台}
  学习\textbf{直连交易平台、商品技术框架}相关知识,负责交易和商品搜索相关的Java后端开发工作
  \begin{spacing}{1.1} 
  \begin{itemize}
    \item 端午节线上问题排查处理,通过\textbf{分析线上日志},定位到问题所在。通过修复程序bug,处理标准商品交易链路中的错误码和外部接口返回解析不一致的问题,解决了线上问题。
    \item 研究并实现商品框架的直连订单批量检索功能,通过\textbf{线程池}优化多条件订单搜索性能。
    \item 从供应商接口获取酒店信息,通过分析酒店信息的数据结构,实现酒店图片的\textbf{批量保存和房型描述信息的解析和存储}。
  \end{itemize}
  \end{spacing} 

  \datedsubsection{\textbf{微软(亚洲)互联网工程院}}{2022年11月 -- 2023年6月}
  \role{Bing数据/软件开发实习生(Data/Software Develop Engineer Intern)}{STCA Beijing Bing}
  \begin{spacing}{1.1} 
  1. \textbf{Microsoft Creator Copilot} 
  \\
  基于Bing搜索引擎和ChatGPT模型的\textbf{AIGC自动创作工具},提供段落实时改写、热搜话题新闻生成以及自动配图等功能。
  \begin{itemize}
    % ReAct框架将对话系统的任务分解为两种类型的动作:推理动作(Reasoning Actions)和搜索动作(Search Actions)。推理动作是指LLM根据当前的对话状态和用户输入,生成一个或多个候选的回答或询问。搜索动作是指LLM根据当前的对话状态和用户输入,生成一个或多个候选的搜索查询,并利用搜索引擎获取相关的搜索结果。
    \item 后端开发:我主要负责设计和实现写作工具的\textbf{后端架构},采用\textbf{代理模式}包括搜索、LLM模型的\textbf{代理层}和\textbf{业务层}。代理层负责与\textbf{搜索引擎和LLM模型服务接口进行交互},实现搜索、模型的调用和结果返回。业务层负责处理用户请求和响应,实现写作工具的核心功能,定制优化写作工具的功能。为了提高框架的性能和内容生成体验,我在代理层采用了\textbf{接口响应优化、HTTP连接池、线程池、Redis缓存}等技术,有效地提升了并发请求的吞吐量和响应速度。
    \item Prompt工程:采用LLM ReAct(Reasoning and Acting)范式,结合\textbf{推理}动作和\textbf{搜索}动作功能使ChatGPT模型可以与Bing搜索引擎交互,获取\textbf{实时热搜信息和搜索结果信息}作为知识库,克服了思维链推理中普遍存在的妄想和错误传播问题,并生成更合理的类人任务解决轨迹,提高大语言模型AI新闻创作的真实性和创造性。
  \end{itemize}
  2. \textbf{Bing用户增长数据分析、搜索平台工具研发}
  \begin{itemize}
    \item 负责Bing Search中\textbf{用户搜索相关的数据开发和数据分析},使用微软Cosmos数据平台Scope语言(类Spark SQL)对搜索日志实时数仓相关数据进行查询、整合,计算New Bing相关的用户增长指标进行分析。通过改进本土化搜索功能提高DAU、DSQ等指标,并通过ABTest进行功能分析。
    \item 维护\textbf{Bing Search}的平台工具,负责\textbf{Bing中国区首页新闻热点数据的实时更新}功能。从对象存储中读取数据,通过Scope查询语言对热搜数据的提取、加工、转换、实现了Bing HotSearch热搜、Bing新标签页股票指数更新等功能,推动了Bing首页界面显示的优化。
  \end{itemize}
  \end{spacing}
  
  \datedsubsection{\textbf{上海哔哩哔哩科技有限公司}}{2022年6月 -- 2022年9月}
  \role{数据平台Java后端开发实习生}{bilibili 数据平台部 Berserker}
  \begin{spacing}{1.1} 
  Berserker是提供数据的存储、查询、数据开发、数据质量分析、分布式调度的大数据平台。
  实习中负责工具侧\textbf{元数据、数据运营、数据管理}等方向,专注\textbf{元数据采集、治理工具}等功能的研发。
  \begin{itemize}
    \item 设计了Hive MetaStore中\textbf{分区信息数据的解耦方案}。通过\textbf{全量拷贝、Binlog增量变更、数据一致性检测、分布式调度同步}的方法落入\textbf{TiDB}的方案加速查询,优化了Hive表分区信息查询速度和资源占用,\textbf{接口性能提升60\%}
    \item 在部门降本增效计划中,参与了无效数据表集中下线的功能设计,为业务提供打标,批量定时删除并在删除后及时告知表所有人的功能,通过\textbf{分布式定时调度框架}设计实现了高效的数据表安全删除逻辑功能
    \item 为了提高数据访问的性能,需要区分\textbf{冷热数据}。同时,为了支持各类数据应用和数据治理的需求,需要提供表使用热度的统计和查询功能,在数据库层面统计各个表的使用情况,设计了覆盖全、数据准、粒度细的\textbf{表使用热度}功能服务,为各类数据应用和数据治理提供支持。
  \end{itemize}
\end{spacing} 
    % 在大数据场景下,数据量庞大,存储成本高,缓存利用率低。为了提高数据访问的效率和性能,需要区分冷热数据,将热数据缓存到高速存储介质上,将冷数据迁移到低成本存储介质上。同时,为了支持各类数据应用和数据治理的需求,需要提供表使用热度的统计和查询功能。

    % 业务结果:通过表使用热度功能服务,可以实现以下目标:
    
    % 节约整体的存储成本,降低冷数据的占用空间。
    % 提高缓存利用效率,加速热数据的访问速度。
    % 支持各类数据应用和数据治理的分析和决策,如优化表结构、调整分区策略、制定归档策略等。
    %     1. 数据库层面的数据统计
% 首先,您需要在数据库层面统计各个表的使用情况。将各类数据的使用情况包括应用访问次数、访问时长、操作类型、查询条件等信息统计到数据库表中。这个数据库表可以设计成包含数据表ID、表名、应用名称、数据访问次数、访问时长、最近访问时间等字段。

% 2. 统计分析
% 对于每张表所记录的访问信息,您需要进行统计分析,并对结果进行存储。您可以使用spark等大数据框架处理这些访问数据,包括按访问次数或访问时长排序,统计每个应用对每个表的访问情况等。具体操作可以根据业务需求和数据量大小进行选择。

% 3. 热度服务接口的实现
% 将前两个步骤统计和分析的结果封装成热度服务接口,提供给各类数据应用和数据治理使用。该接口可以设计为Restful风格,支持各种访问方式,比如查询某个数据表的热度,查询某个应用对某张数据表的访问情况等。接口返回内容可以根据业务需求进行设计,比如返回数据表的名字、使用次数和使用时长等维度的指标。

% 4. 热度展示信息的呈现
% 最后,您需要将热度服务接口的结果进行呈现。可以在数据平台的首页、数据目录、应用界面等位置展示表的热度信息、趋势图、排行榜等。同时也可以通过热度服务提供给数据治理使用,比如对使用次数较少的表进行删除、对使用靠前的表进行备份、对数据质量较差的表进行整理等。

% 总结起来,通过对数据库层面的数据统计、分析和封装成服务接口,实现数据热度的收集和展示,这有助于提升大数据平台的数据应用价值,优化数据治理等业务操作,从而进一步提高数据质量和增强数据资产价值。



  % \section{\faUsers\ 项目经历}
  % \datedsubsection{\textbf{用于列车售票的可线性化并发数据结构}}{2021年9月 -- 2021年12月}
  % \role{个人项目}{中国科学院大学《并发数据结构与多核编程》,林惠民院士}
  % \begin{spacing}{1.2}
  %   使用Java语言设计并完成了一个\textbf{用于列车售票的可线性化并发数据结构}
  %   \begin{itemize} 
  %     \item 支持查票、购票、退票等操作,同时保证每个操作都能\textbf{线性化}地执行,即在并发环境下,每个操作都能按照某个全局顺序执行,不会出现不一致或冲突的情况。 
  %     \item 采用乘车区间\textbf{二进制编码运算}的算法,将每个区间用一个二进制位表示,从而将锁的粒度从区间级别降低到位级别,大大减少了锁的竞争和开销。同时,基于\textbf{CAS原语}(比较并交换),实现了\textbf{lock-free}(无锁)的并发方案,避免了死锁、饥饿等问题,提高了并发效率。 
  %     \item 引入余票表缓存,在购票、退票后多线程异步刷新余票表,使得查票操作可以直接从缓存中读取数据,提高了查询速度;同时通过随机占座等负载均衡的优化,使得不同线程尽量访问不同的资源,避免了资源竞争,提高了并发性能。 
  %     \item 按照70\%查票、20\%购票、10\%退票的概率进行压力测试,利用吞吐量、操作时延等性能评价指标,对算法性能进行评估,最终性能评测分数排名达到所有方案的\textbf{前5\%}。并且使用可线性化验证工具对并发数据结构进行可线性化分析验证,证明了算法的正确性。
  %     \item 使用日志系统进行容灾处理,即每次执行操作时都会记录操作的类型、参数、结果等信息到一个日志文件中。这样可以保证在断电后可以根据日志文件重放操作,并且恢复到最新的状态。
  %    \end{itemize} 
  % \end{spacing}


  \section{\faUsers\ 个人项目} 
  \datedsubsection{\textbf{使用Java语言设计的用于火车票销售的可线性化并发数据结构}}{2021年9月 - 2021年12月} 
  \role{课程设计}{中国科学院大学《并发数据结构与多核编程》,林惠民院士} 
  \begin{spacing}{1.1} 在本项目中,我使用Java语言设计并实现了一个\textbf{基于lock-free算法的高效列车售票系统},该系统支持查票、购票、退票等操作,并能在并发环境下保证每个操作都能\textbf{线性化}地执行,即按照某个全局顺序执行,不会出现不一致或冲突的情况。具体来说,我采用了以下技术和优化: 
    \begin{itemize} 
      \item 利用乘车区间\textbf{二进制编码运算}的技术,将每个区间用一个二进制位表示,从而将锁的粒度从区间级别降低到位级别,大大减少了锁的竞争和开销。 
      \item 基于\textbf{CAS原语}(比较并交换),实现了\textbf{lock-free}(无锁)的并发方案,避免了死锁、饥饿等问题,提高了并发效率。 
      \item 引入余票表\textbf{缓存},在购票、退票后子线程异步刷新余票表,使得查票操作可以直接从缓存中读取数据,提高了查询速度;通过随机占座等\textbf{负载均衡}的优化,使得不同线程尽量访问不同的资源,避免了资源竞争,提高了并发性能。 
      \item 使用日志系统进行\textbf{容灾处理},即每次执行操作时都会记录操作的类型、参数、结果等信息到一个日志文件中。这样可以保证在断电后可以根据日志文件重放操作,并且恢复到最新的状态。 
      \item 按照70\%查票、20\%购票、10\%退票的概率进行压力测试,利用\textbf{吞吐量、操作时延}等性能评价指标,对算法性能进行评估,最终性能评测分数排名达到所有方案的\textbf{前5\%}。并且使用可线性化验证工具对并发数据结构进行可线性化分析验证,证明了算法的正确性。 
    \end{itemize} 
  \end{spacing}


% \datedsubsection{\textbf{中铁隧道局集团有限公司}}{2019 年9月 -- 2019 年 12 月}
% \role{移动端负责人}{中铁隧道局机况检测系统}
% \begin{spacing}{1.2}
% 移动端设备检测模块Android项目负责人,协调后端开发
% \begin{itemize}
%   \item 实现了隧道局盾构机机况检测App,设计了\textbf{图像压缩、数据可靠传输、系统安全性保障}的方案
%   \item 参与了基于SpringBoot框架的中铁隧道局管理系统后端开发,负责\textbf{用户权限管理}和\textbf{安全性校验}的设计。
%   \item 负责项目部署和企业工作人员对接,此项目\textbf{已上线}应用于中铁隧道局集团有限公司日常的盾构机机况检测工作中。
%   \item 涉及技术:Java、SpringBoot、Android
% \end{itemize}
% \end{spacing}

% Reference Test
%\datedsubsection{\textbf{Paper Title\cite{zaharia2012resilient}}}{May. 2015}
%An xxx optimized for xxx\cite{verma2015large}
%\begin{itemize}
%  \item main contribution
%\end{itemize}

\section{\faHeartO\ 获奖情况}
\begin{itemize}[parsep=0.5ex]
  \item 中国科学院大学三好学生
  \item 国家励志奖学金
  \item 中国科学院大学计算机科学与技术学院优秀学生
  \item 全国大学生数学建模竞赛二等奖
  \item 全国大学生数学竞赛三等奖
\end{itemize}

\section{\faCogs\ 额外技能}
\begin{itemize}[parsep=0.5ex]
  \item 英语 :		   CET 4 566、CET 6 479
  \item 爱好 :		   乒乓球、羽毛球
  \item 文档 :		   Markdown、\LaTeX
\end{itemize}


% \section{\faInfo\ 其他}
% % increase linespacing [parsep=0.5ex]
% \begin{itemize}[parsep=0.5ex]
%   \item GitHub: https://github.com/username
%   \item 语言: 英语 - 熟练(TOEFL xxx)
% \end{itemize}

%% Reference
%\newpage
%\bibliographystyle{IEEETran}
%\bibliography{mycite}
\end{document}
