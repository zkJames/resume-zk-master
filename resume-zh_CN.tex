% !TEX TS-program = xelatex
% !TEX encoding = UTF-8 Unicode
% !Mode:: "TeX:UTF-8"

\documentclass{resume}
\usepackage{zh_CN-Adobefonts_external} % Simplified Chinese Support using external fonts (./fonts/zh_CN-Adobe/)
% \usepackage{NotoSansSC_external}
% \usepackage{NotoSerifCJKsc_external}
% \usepackage{zh_CN-Adobefonts_internal} % Simplified Chinese Support using system fonts
\usepackage{linespacing_fix} % disable extra space before next section
\usepackage{cite}

\begin{document}
\pagenumbering{gobble} % suppress displaying page number

\name{朱奎\quad James}

\basicInfo{
  \email{zhukui.1998@qq.com} \textperiodcentered\ 
  \phone{(+86) 131-2671-6166} \textperiodcentered\ 
  \github{https://github.com/zkJames}\
}
  % \linkedin[billryan8]{https://www.linkedin.com/in/billryan8}}
  \section{\faGraduationCap\ 教育背景}
  \datedsubsection{\textbf{中国科学院大学},计算机软件与理论}{2021.9 - 预计2024.6毕业}
  \begin{itemize}
    \item \textit{学术硕士在读(保研)}
    \item \textbf{GPA:} 3.5 / 4.0\space \space \textbf{排名:} 4 /  63 (前10\%)
  \end{itemize}
  \datedsubsection{\textbf{河南科技大学},计算机科学与技术}{2017.9 - 2021.6}
  \begin{itemize}
    \item \textit{本科}
    \item \textbf{GPA:} 4.5 / 5.0\space \space \textbf{排名:} 3 / 144 (前3\%)
  \end{itemize}

\section{\faUsers\ 项目/实习经历}

\datedsubsection{\textbf{微软(亚洲)互联网工程院}}{2022年11月 -- 至今}
\role{Software Engineering Intern}{STCA BJ Bing}
目前负责\textbf{Creator Copilot}项目的后端功能研发,基于ChatGPT模型和实时热搜数据实现\textbf{自动创作工具}。实习中还参与了\textbf{Bing用户增长数据分析、搜索平台工具研发}等工作。
\begin{itemize}
  \item 负责\textbf{ChatGPT和DALLE-2模型的工程化},实现基于Bing HotSearch搜索的内容自动创作工具。主要负责后端开发,实习中根据业务进行查询优化、数据存储优化。
  \item 负责Bing Search中\textbf{用户搜索相关的数据开发},对New Bing相关的用户增长相关进行分析,并通过改进搜索功能提高相关指标。
  \item 维护和开发\textbf{Bing Search}的平台工具,负责\textbf{Bing中国区首页新闻热点数据的实时更新}功能。实现了Bing HotSearch热搜、Bing新标签页股票指数更新等功能。
  \item 涉及技术:C\#、ASP.NET、Scope(Sql)
\end{itemize}

\datedsubsection{\textbf{上海哔哩哔哩科技有限公司}}{2022年6月 -- 2022年9月}
\role{大数据平台研发实习生}{bilibili 数据平台部 Berserker}
Berserker是提供数据的存储、查询、数据开发、数据质量分析、分布式调度的大数据平台。\\
实习中负责工具侧\textbf{元数据、数据运营、数据管理}等方向,专注\textbf{元数据采集、治理工具}等功能的研发。
\begin{itemize}
  \item 设计了Hive MetaStore中\textbf{分区信息数据的解耦方案}。通过\textbf{全量拷贝、Binlog增量变更、数据一致性检测、分布式调度同步}的方法落入\textbf{TiDB}的方案加速查询,优化了Hive表分区信息查询速度和资源占用,\textbf{接口性能提升60\%}
  \item 在部门降本增效计划中,参与了无效数据表集中下线的功能设计,通过\textbf{分布式定时调度框架Archer}设计实现了高效的数据表安全删除逻辑功能
  \item 设计了覆盖全、数据准、粒度细的\textbf{表使用热度}功能服务,为各类数据应用和数据治理提供支持。
  \item 涉及技术:Java、Kafka、SpringBoot、Hive、TiDB、RocketMQ、ClickHouse、MySQL
\end{itemize}

\datedsubsection{\textbf{用于列车售票的可线性化并发数据结构}}{2021年9月 -- 2021年12月}
\role{个人项目}{中国科学院大学《并发数据结构与多核编程》,林惠民院士}
\begin{onehalfspacing}
  使用Java语言设计并完成了一个\textbf{用于列车售票的可线性化并发数据结构}
\begin{itemize}
  \item 设计乘车区间\textbf{二进制编码运算}的算法,基于\textbf{CAS原语}实现\textbf{lock-free}级别的并发方案,性能上大幅度改进。并且设计采用可线性化验证工具对并发数据结构进行可线性化分析验证。
  \item 利用性能评价指标,对算法性能进行评估,\textbf{最终性能评测分数排名}达到所有方案的\textbf{前5\%}
  \item 涉及技术:Java、并发编程、lock-free
\end{itemize}
\end{onehalfspacing}

% \datedsubsection{\textbf{中铁隧道局集团有限公司}}{2019 年9月 -- 2019 年 12 月}
% \role{移动端负责人}{中铁隧道局机况检测系统}
% \begin{onehalfspacing}
% 移动端设备检测模块Android项目负责人,协调后端开发
% \begin{itemize}
%   \item 实现了隧道局盾构机机况检测App,设计了\textbf{图像压缩、数据可靠传输、系统安全性保障}的方案
%   \item 参与了基于SpringBoot框架的中铁隧道局管理系统后端开发,负责\textbf{用户权限管理}和\textbf{安全性校验}的设计。
%   \item 负责项目部署和企业工作人员对接,此项目\textbf{已上线}应用于中铁隧道局集团有限公司日常的盾构机机况检测工作中。
%   \item 涉及技术:Java、SpringBoot、Android
% \end{itemize}
% \end{onehalfspacing}

% Reference Test
%\datedsubsection{\textbf{Paper Title\cite{zaharia2012resilient}}}{May. 2015}
%An xxx optimized for xxx\cite{verma2015large}
%\begin{itemize}
%  \item main contribution
%\end{itemize}

\section{\faHeartO\ 获奖情况}
\begin{itemize}[parsep=0.5ex]
  \item 中国科学院大学三好学生
  \item 国家励志奖学金
  \item 中国科学院大学计算机科学与技术学院优秀学生
  \item 全国大学生数学建模竞赛二等奖
  \item 全国大学生数学竞赛三等奖
\end{itemize}

\section{\faCogs\ 技能清单}
% increase linespacing [parsep=0.5ex]
\begin{itemize}[parsep=0.5ex]
  \item 英语 :		   CET 4 566、\textbf{CET 6 479}
  \item 编程 :		   Java、C/C++、Python、C\#
  \item 文档 :		   Markdown、LaTex
  \item CI/CD:		  Git、GiHub、Docker
  \item 数据库:		   MySQL、TiDB、Hive、SQL Server
  
\end{itemize}


% \section{\faInfo\ 其他}
% % increase linespacing [parsep=0.5ex]
% \begin{itemize}[parsep=0.5ex]
%   \item GitHub: https://github.com/username
%   \item 语言: 英语 - 熟练(TOEFL xxx)
% \end{itemize}

%% Reference
%\newpage
%\bibliographystyle{IEEETran}
%\bibliography{mycite}
\end{document}
